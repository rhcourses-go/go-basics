\def\exampletable{
    \begin{tabularx}{\textwidth}{r|X}
        \textbf{Berechnung} & \textbf{Zwischenergebnis} \\
        \hline
        $1$             & $1$ \\
        $2 \cdot 1$     & $2$ \\
        $3 \cdot 2$     & $6$ \\
        $4 \cdot 6$     & $24$ \\
        $5 \cdot 24$    & $120$ \\
    \end{tabularx}
}

\def\explanationfile{\examplesdir/factorial/explanation.go}
\newcommand{\stepwisefactorial}{\inputsrcfile[linerange={5-9}]{\explanationfile}}
\newcommand{\loopfactorial}{\inputsrcfile[linerange={16-19}]{\explanationfile}}
\newcommand{\loopgeneralized}{\inputsrcfile[linerange={26-29}]{\explanationfile}}
\newcommand{\factorialfunction}{\inputsrcfile[linerange={25-32}]{\explanationfile}}
\newcommand{\backwardsfunction}{\inputsrcfile[linerange={36-43}]{\explanationfile}}
\newcommand{\recursive}{\inputsrcfile[linerange={47-52}]{\explanationfile}}

\begin{fframe}
    \begin{block}{Ziel: Berechne $5!$ }
        \begin{itemize}
            \item Es gilt: $5! = 1 \cdot 2 \cdot 3 \cdot 4 \cdot 5 = 120$
            \item<2-> Kann schrittweise mit Zwischenergebnissen berechnet werden:
                      \exampletable
            \item<3-> So ähnlich würde man es auf Papier berechnen.
            \item<3-> Ziel: Automatisiere die Berechnung.
        \end{itemize}
    \end{block}
\end{fframe}

\begin{fframe}
    \begin{block}{Umsetzung der Schritt-Für-Schritt-Berechnung}
        \stepwisefactorial
        \vspace{.5ex}
        \onslide<2->{\exampletable}
    \end{block}
\end{fframe}

\begin{fframe}
    \begin{block}{Umsetzung der Schritt-Für-Schritt-Berechnung}
        \stepwisefactorial
    \end{block}
    \begin{itemize}
        \item Problem: Die Berechnung ist sehr starr.
        \item Umständlich aufzuschreiben und anzupassen.
        \item<2-> Lösung: Schleifen
    \end{itemize}
\end{fframe}

\begin{fframe}
    \begin{block}{Schrittweise Berechnung wie zuvor}
        \stepwisefactorial
    \end{block}
    \begin{block}<2->{Berechnung mit Schleife}
        \loopfactorial
    \end{block}
\end{fframe}

\begin{fframe}
    \begin{block}{Berechnung mit Schleife}
        \loopfactorial
    \end{block}
    \begin{block}<2->{Vorteile:}
        \begin{itemize}
            \item kompakterer Code
            \item Nur an einer Stelle ändern, um $n$ zu ändern.
            \item Nächster Schritt: $n$ durch eine Variable ersetzen.
        \end{itemize}
    \end{block}
\end{fframe}

\begin{fframe}
    \begin{block}{Berechnung von $5!$}
        \loopfactorial
    \end{block}
    \begin{block}<2->{Berechnung von $n!$}
        \loopgeneralized
    \end{block}
\end{fframe}

\begin{fframe}
    \begin{block}{Berechnung von $n!$}
        \loopgeneralized
    \end{block}
    \begin{block}<2->{Vorteile:}
        \begin{itemize}
            \item Flexibel, $n$ kann z.B. eingelesen oder berechnet werden.
        \end{itemize}
    \end{block}
    \begin{block}<3->{Nachteile:}
        \begin{itemize}
            \item Code kann noch nicht wiederverwendet werden.
            \item Muss ggf. an mehrere Stellen kopiert werden.
            \item<4-> Nächster Schritt: Funktionen
        \end{itemize}
    \end{block}
\end{fframe}

\begin{fframe}
    \begin{block}{Berechnung von $n!$}
        \factorialfunction
    \end{block}
    \begin{block}<2->{Beobachtungen:}
        \begin{itemize}
            \item Code ist in einer \alert{Funktion} \emph{eingepackt}.
            \item Die Funktion kann an anderer Stelle verwendet werden.
        \end{itemize}
    \end{block}
\end{fframe}

\begin{fframe}
    \begin{block}{Alternative: Rückwärts laufende Schleife}
        \backwardsfunction
    \end{block}
    \begin{block}<2->{Beobachtungen:}
        \begin{itemize}
            \item Die Schleife hat einen \alert{Zähler} und eine \alert{Abbruchbedingung}.
            \item<3-> \alert{Eines der wichtigsten Konzepte in der Programmierung!}
        \end{itemize}
    \end{block}
\end{fframe}

\begin{fframe}
    \begin{block}{Alternative: Rekursive Berechnung}
        \recursive
    \end{block}
    \begin{block}<2->{Basiert auf folgender Beobachtung:}
        \vspace{-3ex}
        \begin{align*}
            n! &= n \cdot \text{\colorbox{red!10!white}{$(n-1) \cdot (n-2) \cdots 2 \cdot 1$}} \\
               &= n \cdot \text{\colorbox{red!10!white}{$(n-1)!$}}
        \end{align*}
    \end{block}
\end{fframe}
