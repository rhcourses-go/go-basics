\newcommand{\helloworldsrcblock}{
    \begin{block}{Das erste Programm}
        \inputsrcfile{\examplesdir/helloworld/helloworld.go}
    \end{block}
}

\begin{fframe}
    \helloworldsrcblock
    \begin{block}<2->{Zeile 1: Definition des Pakets, zu dem die Datei gehört.}
        \begin{itemize}
            \item Jedes Programm gehört zu einem \alert{Paket}.
            \item Dient zur Strukturierung von komplexerem Code.
        \end{itemize}
    \end{block}
\end{fframe}
\begin{fframe}
    \helloworldsrcblock
    \begin{block}{Zeile 3: Import-Statement}
        \begin{itemize}
            \item Importiert ein anderes Paket.
                  (Hier: \texttt{fmt} für \emph{format}).
            \item Wird für die Ausgabe benötigt.
        \end{itemize}
    \end{block}
\end{fframe}
\begin{fframe}
    \helloworldsrcblock
    \begin{block}{ab Zeile 5: \texttt{main}-Funktion}
        \begin{itemize}
            \item Jedes Programm muss eine \texttt{main}-Funktion enthalten.
            \item Wird beim Start des Programms ausgeführt.
        \end{itemize}
    \end{block}
\end{fframe}
\begin{fframe}
    \helloworldsrcblock
    \begin{block}{Zeile 6: Ausgabe}
        \begin{itemize}
            \item \texttt{fmt.Println} gibt aus, was in den Klammern steht.
            \item \texttt{fmt} ist ein Paketname,
                  \texttt{Println} eine \alert{Funktion}.
        \end{itemize}
    \end{block}
\end{fframe}
