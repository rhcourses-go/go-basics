\newcommand{\simpleinoutsrcblock}{
    \begin{block}{Einlesen von Benutzereingaben}
        \inputsrcfile[linerange={1-8,14-15}][]{\examplesdir/userinput/userinput.go}
    \end{block}
}

\newcommand{\advancedinoutsrcblock}{
    \begin{block}{... mit Überprüfung der Eingabe.}
        \inputsrcfile[linerange={5-15}]{\examplesdir/userinput/userinput.go}
    \end{block}
}

\begin{fframe}
    \begin{block}{Wichtiger Aspekt: Interaktion mit dem Benutzer}
        \begin{itemize}
            \item Geschieht über das Package \alert{\texttt{fmt}}.
            \begin{itemize}
                \item<2-> \texttt{fmt} steht für \emph{format}.
                \item<2-> Bietet Funktionen zum Einlesen und Ausgeben von Daten.
            \end{itemize}
            \item<3-> Schon bekannt: \texttt{fmt.Println()}.
            \item<4-> \texttt{fmt.Scan()} liest eine Eingabe ein.
        \end{itemize}
    \end{block}
\end{fframe}

\begin{fframe}
    \simpleinoutsrcblock
\end{fframe}

\begin{fframe}
    \advancedinoutsrcblock
\end{fframe}
