\newcommand{\intvariablessrcblock}{
    \begin{block}{Integer-Variablen}
        \inputsrcfile[linerange={5-11}]{\examplesdir/concepts/variables.go}
    \end{block}
}

\newcommand{\stringvariablessrcblock}{
    \begin{block}{String-Variablen}
        \inputsrcfile[linerange={13-20}]{\examplesdir/concepts/variables.go}
    \end{block}
}

\newcommand{\listvariablessrcblock}{
    \begin{block}{Listen-Variablen}
        \inputsrcfile[linerange={22-32}]{\examplesdir/concepts/variables.go}
    \end{block}
}

\begin{fframe}
    \begin{block}{Wichtige Bestandteile von Programmen: \alert{Variablen}}
        \begin{itemize}
            \item<1-> Variablen sind \alert{Speicherplätze} für Werte.
            \item<2-> Müssen \alert{deklariert} werden.
            \item<3-> Anschließend können darin Werte gespeichert werden
                    und man kann mit diesen Werten rechnen.
        \end{itemize}
    \end{block}
    \begin{block}<4->{Technische Sicht}
        \begin{itemize}
            \item<4-> Variablen sind \alert{Speicherbereiche} im \emph{Arbeitsspeicher}.
            \item<5-> Die Größe des Bereichs hängt vom \alert{Typ} der Variable ab.
            \item<6-> Der Typ einer Variable muss bei der Deklaration klar sein.
            \begin{itemize}
                \item<7-> Notwendig, um den Speicher korrekt zu reservieren.
                \item<7-> Nützlich, um das Programm vorab auf Fehler zu überprüfen.
            \end{itemize}
        \end{itemize}
    \end{block}
\end{fframe}

\begin{fframe}
    \intvariablessrcblock
    \begin{itemize}
        \item<2-> Deklaration: \alert{Reservieren von Speicher}
        \item<2-> Rechnen mit den Werten ist möglich.
    \end{itemize}
\end{fframe}

\begin{fframe}
    \stringvariablessrcblock
    \begin{itemize}
        \item<2-> Wie bei Integern, nur der \alert{Typ} ist anders.
        \item<2-> Auch mit Strings kann gerechnet werden.
    \end{itemize}
\end{fframe}

\begin{fframe}
    \listvariablessrcblock
    \begin{itemize}
        \item<2-> Listen sind (theoretisch) unbegrenzt.
    \end{itemize}
\end{fframe}
